\documentclass{article}
\usepackage{graphicx}
\usepackage{natbib}


\begin{document}

    \title{Notes}
    \author{Author's Name}

    \maketitle

    \begin{abstract}
        The abstract text goes here.
    \end{abstract}

    \section{Introduction}

    \begin{itemize}
        \item Time course of pS6K in AA and AA + rapamycin conditions \cite{patursky-polischuk2014}
        \item Rheb activates AMPK and reduces p27 in TSC2 null cells which in turn reduces cdk2 \cite{lacher2010rheb}
        \item Rheb is constitutively active in TSC2 knockout cells \cite{lacher2010rheb}
        \item In TSC2 null cells, down regulating Rheb down regulated mTORC1 and s6k
        \item TSC2 is a GAP for Rheb~\cite{inoki2003}
        \item The more TSC2 in the system the more Rheb that is hydrolysed \cite{inoki2003}
        \item Rheb-GTP is an activator of mTORC1, measured by an increase in S6K and 4EBP phos
        \item The more RhebGTP present the more mTORC1 activation and S6K/4EBP phos \cite{inoki2003}
    \end{itemize}

    \subsection{\cite{Sarbassov2005phosphorylation}}
    \begin{itemize}
        \item mTORC1 phosphorylates Akt at S473
    \end{itemize}


    \subsection{\cite{Sarbassov2005phosphorylation}}
    This is a review
    \begin{itemize}
        \item Amino acids inhibit TSC2
    \end{itemize}

    \subsection{\cite{hinalt2004amino}}
    \begin{itemize}
        \item Insulin and amino acids both stimulate mTORC1 individually and synergize together
        \item Wortmannin inhibits these reactions
    \end{itemize}

    \subsection{\cite{hardie2012ampk}}
    AMPK and energy review
    \begin{itemize}
        \item During muscle contraction, glucose is used to generate ATP. AMPK enhances this process
        \item Muscle glucose uptake is also promoted by insulin, when the fate of glucose is storage as glycogen
        \item GLUT4 mediates the glucose update in both of these situations
        \item GLUT4 are glucose transporters that reside in vesicles near the plasma membrane
        \item GLUT4 containing vesicles need to fuse with the plasma membrane to allow them to transport glucose
        \item This fusion requiers RAB G proteins in their GTP bound state, though RAB G proteins are basally exist in their GDP bound state
        \item RAB G proteins are held in an inactive GDP bound state by RAB-GAP proteins, one of which is called Akt substrate 160 (AS160 / TBC1D4) and TBC1D1, both of which are associated with the GLUT4 containing vesicles
        \item Akt phos AS160 in muscle and adipocytes allowing it to associate with 14-3-3 proteins which leads to disociation from the vesicles
        \item AMPK also phosphorylates TBC1D1 in contracting muscle, also recruits 14-3-3 prroteins and leading to dissociation from vesicles
        \item In both biological contexts, the Rab-GAP dissociation results in the loading of Rab with GTP and fusion of GLUT4 containing vesicles with the membrane
        \item .
    \end{itemize}

    \subsection{\cite{hains2019greb1}}
    \begin{itemize}
        \item GREB1 gene response to E2 treatment via the estrogen receptor
        \item GREB1 is required for hormone dependent proliferation
        \item knock down of GREB1 results in growth arrent
        \item overexpression of GREB1 results in oncogenic senescence
        \item GREB1 regulates PI3K/Akt/mTORC1
        \item Growth arrested BC cells from GREB1 knock down can be rescued by constitutive Akt activation
    \end{itemize}

    \subsection{\cite{miller2010hyperactivity}}
    \begin{itemize}
        \item A mechanism of resistance to endocrine therapies is overexpression of HER2
        \item but only a small portion (around 10\%)
    \end{itemize}

    \subsection{\cite{miller2010hyperactivity}}
    \begin{itemize}
        \item These guys generated 4 long term estrogen deprevation BC cell lines (LTED)
        \item Depriving these cells of estrogen is associated with hyperactivation of IGF/Insulin and EGF signalling
        \item
    \end{itemize}

    \subsection{\cite{miller2011phosphatidylinositol}}
    \begin{itemize}
        \item
    \end{itemize}

    \subsection{\cite{campbell2001phosphatidylinositol}}
    \begin{itemize}
        \item
    \end{itemize}

    \subsection{\cite{liu2018targeting}}
    \begin{itemize}
        \item JakStat, Ras-PKC, NFkB and Kit signalling pathways regulate IDO1
        \item Downstream of IDO1, GCN2 is activated, mTORC1 is inhibited by Tryp deprevation and the aryl hydrocarbon receptor pathway (AhR) is activated, since Kyn is ligand for the receptor
        \item IDO1 is not expressed in most tissues under physiological conditions but constitutively expressed in many types of cancer
        \item IDO1 is activated by diverse inflammatory signals: TNFa, TGFb, PAMPS, DAMPs, PGE2, via nfkb and jak stat pathways
        \item constitutive IDO1 expression is mediated by cyclooxygenase and PGE2 via PKC and PI3K pathways \cite{hennequart2017constitutive}
        \item IDO1 is under the regulation of of Bin1. Bin1 KO results in tumour growth and immune supression via upregulation of Stat1 and nfkb dependent \cite{muller2005inhibition}
        \item Ras regulates IDO1 in cancer
        \item
        \item The depletion of Trp by IDO1 leads to accumulation of uncharged Trp-tRNA, which binds to and activates GCN2 a stress response kinase
        \item GCN2 phos and inhibits eukaryotic initiation factor 2 alpha, attenuating transcription and translation
        \item mTOR is also inhibited by Trp deprevation leading to activation of autophagy (via AMPK?)
        \item Both GCN2 and mTOR effectors of IDO1 are immunesuppressive
        \item AhR is a cytosolic ligand-activated TF. Its most potant ligand is TCDD which upregulates IDO1 expression \cite{vogel2008aryl}. Therefore it might be the case that Kyn binding to AhR also upregulates IDO1.
        \item Since Kyn is also a ligand for AhR, we have a positive feedback \cite{opitz2011endogenous}
        \item AhR is bHLH family
        \item Kyn leads to the translocation of AhR into the nucleus after 1h \cite{opitz2011endogenous}
        \item In the nucleus, the AhR forms a heterodimer with ARNT that interacts with dioxin response elements (DRE) \cite{opitz2011endogenous}
        \item Treg cells are the immunosuppressive entity that are recruited when IDO1 levels are high
        \item
        \item
    \end{itemize}

    \subsection{\cite{metz2012ido}}
    \begin{itemize}
        \item trp deprevation by IDO1 inhibits mTORC1, D-1MT releives this event
        \item Trp deprevation leads to down regulation of PKC theta activity
    \end{itemize}




    \subsection{\cite{dong2000uncharged}}
    \begin{itemize}
        \item GCN2 regulates translation in AA starved cells by phosphoryltaing eIF2.
        \item GCN2 is a general sensor of starvation
        \item
    \end{itemize}

    \subsection{\cite{hao2005uncharged}}
    \begin{itemize}
        \item
    \end{itemize}

    \subsection{\cite{xia2018gcn2}}
    \begin{itemize}
        \item GCN2 is one of 4 known sensors of the integrated stress response (IRS)
        \item It can be activated by the deprevation of a single amino acid
        \item Under context of low AA levels, uncharged tRNA molecules accumulate and bind to GNC2 causing its activation
        \item When activated, GCN2 phosphorylates and inactivates eIF2alpha on Ser51, and reduces translation
    \end{itemize}

    \subsection{\cite{folgiero2016ido1}}
    \begin{itemize}
        \item Rapamycin enhances INFg mediated IDO1 expression
        \item
    \end{itemize}

    \subsection{\cite{fumarola2013effects}}
    \begin{itemize}
        \item Sorafenib is a multi kinase inhibitor
        \item Inhibits Raf
    \end{itemize}

    \subsection{\cite{van2011long}}
    \begin{itemize}
        \item LKB1 activates AMPK during times of food scaricity.
        \item LKB1 resides in the nucleus. On activation, LKB1 forms a heterotrimer with STRAD$\alpha$ and MO25
        \item Strad$\alpha$ prevents nuclear relocalisation of LKB1 while MO25 stablizes the interaction between Strad and LKB1
        \item When cytoplasmic and active, LKB1 activates AMPK
        \item AMPK is a sensor of cellular energy status via ratios of AMP to ATP and ADP to ATP.
        \item When AMPK is bound to AMP (or ADP) it can be phosphorylated by LKB1.
        \item An alternative mode of AMPK activation is via CaMKK2 in response to Ca2+ influx
        \item The reverse reaction of AMPK dephosphorylation is catalysed by PP2A and PP2C.
        \item The consequence of AMPK activation is the production of metabolic enzymes such as acetyl-CoA carboxylase and HMG-CoA reductase.
        \item AMPK is a key component that results in suppression of energy consumption processes such as energy storage in glycogen and lipid synthesis and enhancing energy gaining processes on the other, such as glycolysis.
        \item Therefore AMPK results in restoration of cellular energy status.
        \item Further, AMPK stimulates translocation of GLUT4 to the plasma membrane
        \item Importantly, TSC2 is a downstream target of AMPK
        \item Erk inhibits TSC2
        \item
        \item mTORC1 phosphorylates ATG and ULK1/2, inhibiting autophagy.
    \end{itemize}

    \subsection{\cite{hardie2012ampk}}
    \begin{itemize}
        \item
    \end{itemize}

    \section{Model Features and Mechanisms}
    \subsection{PI3K / mTORC1}
    \begin{itemize}
        \item Stimulation with insulin induces PI3K activation and conversion of PIP2 to PIP3.
        \item Akt and PDK1 both bind to PIP3 allowing PDK1 to phos Akt. Activate Akt inactivates TSC2.
        \item Meanwhile, amino acid stimulation causes activataion of Rag G proteins which stimulate the migration of mTOR from the cytoplasm to the lysosome.
        \item TSC2, when active associates with the inactivated form of Rag which maintains Rheb in the inactive state
        \item When TSC is inhibited by phos from Akt, Rheb is primed with GTP, allowing the phos and activation of lysosomal mTORC1
        \item Activated mTORC1 actuvates both 4EBP1 and S6K by phosphorylation.
        \item Wortmannin, MK2206 and Rapamycin inhibit the PI3K system at PI3K, Akt and all mTORC1 subspecies.
    \end{itemize}

    \subsection{EGF}
    \begin{itemize}
        \item EGF binds EGFR which is a receptor tyrosine kinase. The receptor binds scaffolding proteins such as Sos leading to a fully activated receptor.
        \item The activated receptor primes Ras GTP for activating Raf which activates Mek and Erk by sequential phosphorylations.
        \item Two feedbacks are included: a slow and fast.
        \item The slow feedback is by the de novo transcription of phosphatases such as DUSPs.
        \item The fast feedback is Erk directly phosphorylating and inhibiting Sos
        \item AZD inhibits the Erk system at Mek
    \end{itemize}

    \subsection{Cross-talk between Erk and PI3K systems}

    \subsection{AMPK}
    \begin{itemize}
        \item AMPK is activated in two ways: LKB1 and CaMKK2a
        \item LKB1 resides normally in the nucleus and is activated on a starvation signal. Given the model has no starvation signal, i've made the nuclear localisation of LKB1 'feeding` dependent
        \item Without Feeding being active, the flux tends towards LKB1 which phosphorylates AMPK
        \item Calcium can bind CaMKK2 activating it's ability to phos and activate AMPK.
        \item When active AMPK activates Ulks by phos and increases autophagy.
        \item Moreover, when activated, AMPK activates the GAP activity of TSCs, ensuring inactivity of mTORC1.
    \end{itemize}

    \subsection{IP3}
    \begin{itemize}
        \item PIP2 can either undergo conversion to PIP3 via PI3K or can be broken down into IP3 and DAG by active PLC.
        \item DAG goes on to activate PKC while IP3 binds the IP receptor leading to intracellular influx of Ca2+
        \item the ca2+ can then go on to activate AMPK and autophagy
    \end{itemize}

    \subsection{INFg and IDO1}
    \begin{itemize}
        \item INFg activates Jak and Stat, causing cytoplasmic dimerisation of pStat1 followed by nuclear localisation
        \item Nuclear pStat1 dimers regulate transcription, for example of IDO1 mRNA.
        \item The Translation factor eIFa explicitely mediates translation because it is inactivated by phosphorylation.
        \item A lack of Trp caused by too much conversion to Kyn by IDO1 activates GCN2 stress response leading to phos and inactivation of eIFa, thereby inhibiting translation.
    \end{itemize}

    \subsection{Estrogen}
    \begin{itemize}
        \item Estrogen molecules diffuse through membranes due to their lipophillic nature.
        \item Cytosolic E2 binds to ERa in the cytosol which then dimerise
        \item ERa dimers translocate to the nucleus where they can modulate transcription of TFF and Greb1
        \item TFF is a readout of E2 used by the experimentalists
        \item Greb1 is a plausible connection between E2 and PI3K
        \item Fulvesrtrant binds and sequesters ERa in the cytoplasm
    \end{itemize}

    \subsection{Growth}
    It seemed to me that we need to begin thinking a metric for the concept of growth in the model.
    Ideally we'd employ some kind multilevel model model where signalling  can be related to higher level properties such as growth.
    This is harder however so I've put in place a simplisitic growth module

    \begin{itemize}
        \item Proliferation signals are caused by both activated mTORc1 and ERa dimers in the nucleus
        \item Immuno signals are normally high, inhibiting too much growth.
        \item Kyn suppresses the these signals, reducing the immune systems ability to reduce tissue growth.
    \end{itemize}

    %	\subsection{x}
    %%\cite{nobukini2005amino}}
    %	\begin{itemize}
    %		\item Figure 1B
    %		\item S6K response to AA with mild phos but not to insulin. When combined, S6K has much larger phos response.
    %		\item Akt S473 is phos basally with no change with addition of AA. Insulin causes massive phos of Akt but both AA and insulin together produce a weaker signal.
    %		\item TSC2 is not phos by AA stim but is by insulin or both. Looks like AA provides no potentiation to the insulin signal.
    %		\item
    %	\end{itemize}
    %
    %
    %	\cite{Inoki2006}
    %
    %	\section{Papers left to read}
    %	\begin{itemize}
    %		\item Inoki2003 ampk phos tsc2
    %	\end{itemize}
    %
    %
    %\begin{itemize}
    %	\item insulin causes phos of s6k
    %	\item
    %\end{itemize}
    %
    %

    \subsection{\cite{demetriades2014regulation}}
    \begin{itemize}
        \item RAG proteins bind TSC2.
        \item Therefore, RAGs not only activate mTORC1 by inducing their recruitment to the lysosome under AA stimulation, they also actively actively repress mTORC1 in the absence of AA
        \item binding between rag proteins and TSC2 increases upon AA removal
        \item TSC2 is recruited to lysome in a Rag dependent manner when AAs are removed
        \item \begin{itemize}
                  \item TSC is cytoplasmic when AA present
                  \item On removal TSC2 quickly accumulates on lysosomal surface (15 min)
                  \item impaired Rag protein integrity blunts lysosomal accumulation of TSC2 upon AA removal and reduced mtorc activation on AA readdition
        \end{itemize}
        \item TSC2 is required for complete inactivation of mTORC1 when AAs are removed
        \item \begin{itemize}
                  \item cells lacking TSC2are unable to completly inactivate mtor on AA removal
                  \item
                  \item
        \end{itemize}
    \end{itemize}







    \bibliographystyle{unsrt}
    \bibliography{sources}


\end{document}