\documentclass{article}
\usepackage{graphicx}
\usepackage{natbib}


\begin{document}

\title{Introduction to \LaTeX{}}
\author{Author's Name}

\maketitle

\begin{abstract}
The abstract text goes here.
\end{abstract}

\section{Introduction}

	\begin{itemize}
		\item Time course of pS6K in AA and AA + rapamycin conditions \cite{Patursky-Polischuk2014}
		\item Rheb activates AMPK and reduces p27 in TSC2 null  cells which in turn reduces cdk2 \cite{lacher2010rheb}
		\item Rheb is constitutively active in TSC2 knockout cells \cite{lacher2010rheb}
		\item In TSC2 null cells, down regulating Rheb down regulated mTORC1 and s6k
		\item TSC2 is a GAP for Rheb~\cite{Inoki2003}
		\item The more TSC2 in the system the more Rheb that is hydrolysed \cite{Inoki2003}
		\item Rheb-GTP is an activator of mTORC1, measured by an increase in S6K and 4EBP phos
		\item The more RhebGTP present the more mTORC1 activation and S6K/4EBP phos \cite{Inoki2003}
	\end{itemize}

	\subsection{\cite{Sarbassov2005phosphorylation}}
	\begin{itemize}
		\item mTORC1 phosphorylates Akt at S473
	\end{itemize}


	\subsection{\cite{Sarbassov2005phosphorylation}}
	This is a review
	\begin{itemize}
		\item Amino acids inhibit TSC2
	\end{itemize}

	\subsection{\cite{hinalt2004amino}}
	\begin{itemize}
		\item Insulin and amino acids both stimulate mTORC1 individually and synergize together
		\item Wortmannin inhibits these reactions
	\end{itemize}



%
%	\subsection{x}
%%\cite{nobukini2005amino}}
%	\begin{itemize}
%		\item Figure 1B
%		\item S6K response to AA with mild phos but not to insulin. When combined, S6K has much larger phos response.
%		\item Akt S473 is phos basally with no change with addition of AA. Insulin causes massive phos of Akt but both AA and insulin together produce a weaker signal.
%		\item TSC2 is not phos by AA stim but is by insulin or both. Looks like AA provides no potentiation to the insulin signal.
%		\item
%	\end{itemize}
%
%
%	\cite{Inoki2006}
%
%	\section{Papers left to read}
%	\begin{itemize}
%		\item Inoki2003 ampk phos tsc2
%	\end{itemize}
%
%
%\begin{itemize}
%	\item insulin causes phos of s6k
%	\item
%\end{itemize}
%
%

\subsection{\cite{demetriades2014regulation}}
\begin{itemize}
	\item RAG proteins bind TSC2.
	\item Therefore, RAGs not only activate mTORC1 by inducing their recruitment to the lysosome under AA stimulation, they also actively actively repress mTORC1 in the absence of AA
	\item binding between rag proteins and TSC2 increases upon AA removal
	\item TSC2 is recruited to lysome in a Rag dependent manner when AAs are removed
	\item \begin{itemize}
			  \item TSC is cytoplasmic when AA present
			  \item On removal TSC2 quickly accumulates on lysosomal surface (15 min)
			  \item impaired Rag protein integrity blunts lysosomal accumulation of TSC2 upon AA removal and reduced mtorc activation on AA readdition
	\end{itemize}
	\item TSC2 is required for complete inactivation of mTORC1 when AAs are removed
	\item \begin{itemize}
			  \item cells lacking TSC2are unable to completly inactivate mtor on AA removal
			  \item 
			  \item
	\end{itemize}
\end{itemize}







\bibliographystyle{unsrt}
\bibliography{sources}



\end{document}