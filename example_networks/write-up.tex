\documentclass{article}
\usepackage{graphicx}
\usepackage{natbib}


\begin{document}

\title{Introduction to \LaTeX{}}
\author{Author's Name}

\maketitle

\begin{abstract}
The abstract text goes here.
\end{abstract}

\section{Introduction}

	\begin{itemize}
		\item Time course of pS6K in AA and AA + rapamycin conditions \cite{Patursky-Polischuk2014}
		\item Rheb activates AMPK and reduces p27 in TSC2 null  cells which in turn reduces cdk2 \cite{lacher2010rheb}
		\item Rheb is constitutively active in TSC2 knockout cells \cite{lacher2010rheb}
		\item In TSC2 null cells, down regulating Rheb down regulated mTORC1 and s6k
		\item TSC2 is a GAP for Rheb \cite{Inoki2003}
		\item The more TSC2 in the system the more Rheb that is hydrolysed \cite{Inoki2003}
		\item Rheb-GTP is an activator of mTORC1, measured by an increase in S6K and 4EBP phos
		\item The more RhebGTP present the more mTORC1 activation and S6K/4EBP phos \cite{Inoki2003}
	\end{itemize}

	\subsection{\cite{nobukini2005amino}}
	\begin{itemize}
		\item Figure 1B
		\item S6K response to AA with mild phos but not to insulin. When combined, S6K has much larger phos response.
		\item Akt S473 is phos basally with no change with addition of AA. Insulin causes massive phos of Akt but both AA and insulin together produce a weaker signal.
		\item TSC2 is not phos by AA stim but is by insulin or both. Looks like AA provides no potentiation to the insulin signal.
		\item
	\end{itemize}


	\cite{Inoki2006}

	\section{Papers left to read}
	\begin{itemize}
		\item Inoki2003 ampk phos tsc2
	\end{itemize}


\begin{itemize}
	\item insulin causes phos of s6k
	\item
\end{itemize}


\subsection{Ideas for language extension}
Define conditions within the language?
Keywords and language useage for end user:
\begin{itemize}
	\item Oscillations x: Look for oscillations in x
	\item transient increasing x: Look for a transient increasing curve
	\item x@t=5 > x@t=10
	\item max x@t=(0, 100) > 50
	\item
	\item define condition name {Insulin: 1, AA: 0}
	\item define condition combinations name {Insulin: 1, AA: 0}
	\item Then to reference the condition:
	\item all x[name]@t=(0,100) > max x[other_name]@t=(0, 100)
\end{itemize}








\bibliographystyle{unsrt}
\bibliography{sources}



\end{document}